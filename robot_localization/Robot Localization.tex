\documentclass[10pt,a4paper,twocolumn]{article}
\newcommand\tab[1][0.5cm]{\hspace*{#1}}
\title{%
	ROS: Robot Localization \\
	\large Course assignment: Intelligent Agents and Robotic Systems \\
	University of Piraeus, Demokritos}
\author{
	Papadopoulos, Vasileios\\
	\texttt{vassilispapadop@gmail.com}
}

\usepackage[utf8]{inputenc}
\usepackage[T1]{fontenc}
\usepackage{amsmath}
\usepackage{amsfonts}
\usepackage{amssymb}
\usepackage{graphicx}
\usepackage{amsmath}
\usepackage{graphicx}
\usepackage{hyperref}
\graphicspath{ {./images/} }


\begin{document}
\maketitle
	
\section{Introduction}
Robot localization is a process to examine the exact position within its environment; Navigation is a crucial activity for a mobile robot for safety, operation and other reasons. A mobile robot needs to know not only its \textit{absolute} position but also its \textit{relative} position in respect, for example to a human who might interact with it. Navigation process requires 4 building blocks; \textit{Perception}, \textit{Localization}, \textit{Congition} and \textit{Motion Control}.


\section{Methods}
	
\section{Results}

\section{Discussion}

\section{Example}

\begin{thebibliography}{9}
	\bibitem{moore} 
	Thomas Moore and Daniel Stouch. 
	\textit{A Generalized Extended Kalman Filter Implementation for the Robot Operating System}. 
	Sensor Processing and Networking Division Charles River Analytics, Inc. Cambridge, Massachusetts, USA.	
	
\end{thebibliography}


	
\end{document}